Modern high voltage circuit breaker (CB) consists of two contact sets of main and arcing contacts. The main contacts are made of copper with silver coating and carry the normal current continuously without heating whereas tungsten-copper arcing contacts opens followed by main contacts and are exposed to arcing. The arcing contacts have to sustain the arc current, arc energy and the high temperature around 10,000\textdegree K. Hence arcing contacts are liable to damage due to severe thermal stresses and Transient Recovery Voltage (TRV). Damaged main and arcing contacts reduce the short circuit capacity of circuit breaker. Therefore the condition assessment of circuit breakers contact is of prime importance. Static contact resistance measurement evaluates the condition of main contacts only. The lack of direct access to the arcing contacts and use of high pressure gas complicate the direct condition assessment of this part. Hence the Dynamic Contact Resistance Measurement (DCRM) test has been recently introduced as a condition assessment test for CB main and arcing contacts. Many parameters such as length of arcing contact, contact wipe and erosion of main and arcing contacts, contact misalignments, healthiness of linkage mechanism, main and arcing contact resistance, contact travel and speed \textit{etc}. can be obtained from signature.

Switching under different fault condition and certain normal duty leads to severe TRVs across the contacts of the circuit breakers which may fail the circuit breaker to clear the fault and has the influence on the short circuit capacity of the circuit breaker. In this thesis study of TRV under different fault conditions for IEEE network is carried out and the short circuit capability of the CB is determined using computer simulations in EMTP-RV. DCRM tests and timing measurement tests are conducted on 400 kV and 245 kV SF\textsubscript{6} CBs at circuit breaker manufacturing industry, 400 kV substation Waluj, Aurangabad and 765 kV substation at Thapti Tanda Aurangabad. Data of DCRM for healthy breakers as well as breakers with problems in condition of contact was collected from the largest utility company Power Grid Corporation of India Ltd. (PGCIL) as well as Maharashtra State Electricity Transmission Company Ltd. (MSETCL). DCRM signature obtained from the test is difficult to analyze as it needs the knowledge of circuit breaker design, operating mechanism, and interrupter assembly and also expertise to conclude the cause. Numbers of case studies are presented and the measured and collected DCRM data is analyzed and a new algorithm has been proposed to detect the contact anomaly. Computer program is developed to determine the health of CB.
