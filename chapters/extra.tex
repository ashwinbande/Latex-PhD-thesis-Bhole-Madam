Dielectric breakdown		&\multirow{4}{*}{9.9}	& Breakdown to earth: 5\%							\\ 
													&& Internal breakdown across open pole, during opening operation = does 	not break the current: 1.9\% \\
													&& Other across open pole: 1.8\%						\\
													&& Breakdown between poles: 1.2\%					\\ \hline
													
													
\subsubsection{Chapter 1} Introduction - This chapter addresses the aspects containing introductory part. The necessity and the objectives of this research work are clearly mentioned. The overall theme of the complete research work is presented.
\subsubsection{Chapter 2} Literature Survey - A summary of the exhaustive literature survey is represented. An overview of arc modeling, Transient Recovery Voltage, Dynamic Contact Resistance Measurement (DCRM), Standards related to circuit breaker timings are discussed in detail.
\subsubsection{Chapter 3} System Development - Computational model of IEEE network for TRV study under different fault conditions is developed in EMTP-RV. Similarly analytical and mathematical models are developed. Dynamic contact resistance measurement is explained. Mathematical treatment is explained in detail with relevant references.
\subsubsection{Chapter 4} Performance Analysis – Results of analytical and computational methods are presented. Justification for difference is given. Measured and collected data of DCRM from field is analyzed in detail using HISAC ULTIMA test manager software and new algorithm is proposed to detect the contact anomaly. Computer program in Java is developed to determine the health of circuit breaker.
\subsubsection{Chapter 5} Conclusions - In this chapter conclusions of the research work, future work and  applications are presented.


\begin{table}[!htbp]
\begin{threeparttable}
\renewcommand{\arraystretch}{1.3}
\caption{Percentage of Maf Rate and Mif Rate per Failure Mode, Third Inquiry.}
\label{table:Percentage of Maf Rate}
\centering
\small

\begin{tabular}{| l | c | l | l | l | l |} \hline
\multicolumn{2}{|c|}{1} & 2 & 3 & 4 & 5 \\ \hline
\multirow{2}{*}{1}		& 2 & 3 & 4 & 5 & 6 \\ \cline{2-6}
				 		& 2 & 3 & 4 & 5 & 6 \\ \hline
				 		
\multirow{4}{*}{1}		& 2 & 3 & 4 & 5 & 6 \\ \cline{2-6}
						& 2 & 3 & 4 & 5 & 6 \\ \cline{2-6}
						& 2 & 3 & 4 & 5 & 6 \\ \cline{2-6}
				 		& 2 & 3 & 4 & 5 & 6 \\ \hline
\end{tabular}
\end{threeparttable}
\end{table}

\\
%    \begin{subfigure}[b]{0.3\textwidth}
%        \centering
%        \begin{minipage}{\textwidth}
%		\centering
%		\begin{flushleft}
%		\footnotesize a. Inside porcelain wall of stationary main contact is seen with white gray powder. Arcing led to decomposition of SF6 gas forming metal fluorides and metal sulphates.\\
%b. Stationary arc contact is seen with slight burning marks at the tip indicating arcing for substantial times and duration
%		\end{flushleft}
%		\end{minipage}
%    \end{subfigure}
    
    \boxed{\frac{f^2}{1-f^2}P = 3.16 \times 10^{-7} T^{5/2} e^{-e V_i/ kT}}
    
6  \tiny	F-tiny.png
8  \scriptsize	F-scriptsize.png
10 \footnotesize	F-footnotesize.png
12 \small	F-small.png
14 \normalsize	F-normalsize.png
16 \large	F-large.png
18 \Large	F-large2.png
20 \LARGE	F-large3.png
22 \huge	F-huge.png
24 \Huge	F-huge2.png

%@IEEEtranBSTCTL{IEEEexample:BSTcontrol,
%  CTLuse_forced_etal       = "yes",
%  CTLmax_names_forced_etal = "3",
%  CTLnames_show_etal       = "1" 
%}