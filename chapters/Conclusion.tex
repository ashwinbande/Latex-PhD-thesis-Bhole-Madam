%Conclusion
\section{Conclusions}
From the study, simulation results, field measurements and case studies following conclusions are drawn:
\begin{itemize}
\item The TRV across breaker which is used for switching a transformer under three phase to ground terminal fault is much higher than the standard TRV. Higher RRRV may lead to restrike and reignition. Hence either the breaker of higher short circuit interruption capability is to be used, or capacitance should be added to CBs terminals to reduce the RRRV of TRV

\item The TRV across breaker used for multiple line switching is within the standard TRV capability curve. Hence the existing CB can be used for switching

\item For short line fault the magnitude of TRV is quite less but the RRRV is high leading to restrike. Hence either the breaker of higher short circuit interruption capability is to be used or capacitance should be added to CBs terminals to reduce the RRRV of TRV. Constant parameter model of transmission line gives 10.6\% higher magnitude of TRV and 13.49\% higher RRRV compared to frequency dependent model of line

\item Arc interruption studies for a given model show the successful arc interruption for the terminal fault case

\item It is observed from the fault interruption studies that short circuit capability of CB depends upon the type of fault and TRV seen by the breaker

\item Contact travel of CB must be optimally designed. Smaller contact travel will lead to restrike whereas higher value will increase the size of interrupter

\item The trip velocity of CB should be such that assumed arcing time is always less than the total travel time of the interrupter. Low trip velocity will lead to restrike whereas CB with high value of trip velocity would not be able to interrupt the short circuit current

\item Increase in CB main and arcing contact resistance results in increase in TRV reducing the short circuit interruption capability

\item Bouncing in no load curve, current breaking and variations in resistance at the corresponding points in DCRM signature indicates the mechanical problem like lack of tightening torque and wrong assembly of contacts

\item Variations in resistance during the close portion zone of the DCRM signature should be carefully observed

\item Over travel at the end indicates defective damping system

\item DCRM must be recorded with travel transducer and the signature during each maintenance schedule is to be properly recorded for study in the change in signature. Analysis of DCRM signature needs the knowledge of anatomy of CB. That is CB design, operating mechanism to conclude the case

\item DCRM is very useful to the manufacturers as the human error or manufacturing error such as wipe adjustment, lack of tightening torque, wrong assembly of contacts, mechanism problem can be detected before despatch

\item Utility maintenance crew can avoid CB failure and improve the reliability of power system through DCRM
\end{itemize}
 
\clearpage
\section{Future Scope}
This thesis deals with the study of dynamic properties of CB on short circuit capacity. The main focus is on the determination of contact condition of CB which is stressed more during the fault current interruption. The algorithm is proposed for 400 kV and 245 kV SF\textsubscript{6} CB of one make. To conclude the case design details of CB are required. The study can be extended for other make of CBs. The DCRM is included as one of the condition maintenance techniques in PGCIL and few state utility companies. Proper record of DCRM signatures from the time of commissioning can be kept. Post mortem report in case of CB failure will give a strong base for the studies to effectively use the technique of DCRM in future which may give a better understanding of the design of CBs.

\section{Applications}
In the deregulated environment of the fast growing power systems, the reliability of the CB in switching and interrupting the fault current is important. CBs are like insurances for every utility company. If something mishap takes place, then they are there to take care. Hence the condition based maintenance of the CB is of prime importance. Manufacturers can detect any manufacturing error such as wipe requirement, wrong assembly of contacts, lack of tightening torque, mechanism problem which is not detected during the set procedure of testing. Utility engineers can detect the contact anomaly and avoid the major failure of the system. The proposed algorithm is useful to manufacturers as well as utility engineers to detect the problem at an early stage and avoid the major accident and failure of the system.
\clearpage

%-----------------------Contributions------------------------------------------------------

\section{Contributions}
\begin{enumerate}
\item Number of DCRM signatures from the field are analyzed and conclusions are drawn which will be useful in reading the DCRM signature
\item Developed an algorithm to determine the contact condition from DCRM signature that suggests the availability of CB for future switching operation
\item Identified a program to determine the health of CB during performance
\item Short circuit capability of CB under different fault conditions and its suitability for particular switching operation is determined
\item EMTP-RV model for Trnsient Recovery Voltage under faulty condition is
suggested
\end{enumerate}
\clearpage